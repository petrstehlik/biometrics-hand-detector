\documentclass[11pt,a4paper]{article}
\usepackage[left=2cm,text={17cm,25cm},top=2.5cm]{geometry}
\usepackage[T1]{fontenc}
\usepackage[english]{babel}
\usepackage[utf8]{inputenc}
\usepackage{url}
\usepackage{graphicx}
\usepackage{pdfpages}
\usepackage[colorinlistoftodos,prependcaption,textsize=tiny]{todonotes}

\graphicspath{ {figs/} }

\begin{document}

\begin{center}
	\LARGE{Biometric Systems}\\
	\Large{Capturing of 2D Hand Geometry With Line Camera}
	\vspace{0.5cm}

    \begin{centering}
    \small{
        Petr Stehlík <xstehl14@stud.fit.vutbr.cz>\\
        Marek Beňo <xbenom01@stud.fit.vutbr.cz>\\
        Richard Wolfert <xwolfe00@stud.fit.vutbr.cz>\\
        }
    \end{centering}

	\vspace{0.2cm}

\end{center}

\section{Assignment}
The aim of the project is to assemble a mechanical device and create software for acquiring 2D hand geometry via a line camera.
A solution for camera mount and hand placement is proposed in the following sections for best geometry acquisition and image reconstruction.

The document is structured as follows: in section \ref{sec:hw} we describe the hardware part of the project, what hardware was used
and we describe a solution for camera mount and hand placement. In section \ref{sec:sw} we present the software needed for camera control and image reconstruction.
In section \ref{sec:db} we detail created database of hand geometry data and method of acquisition. In last two sections \ref{sec:res}
and \ref{sec:sum} we sum up hardware and software solution and propose further improvements on our solution.

\section{Hardware}
\label{sec:hw}

In this section we describe the hardware we used in the process of creating the device.

\subsection{Hardware Setup}
% TODO: čo bolo treba vytvoriť na sprovoznenie
% zapojenie arduino, raspberry
% popísať prečo je potreba arduino a púrečo je pi zlé (GPIO 3.3V vs 5V)
% schema konstrukcie / foto


\subsubsection*{Camera}
The selected camera is Basler raL6144-16gm. This camera provides us with the resolution of 6144 $\times$ 1 pixels with line rate up to 17 kHz. The captured image is in greyscale colors which for our use case is the desired output. The camera's shutter can be operated either via hardware or software trigger\footnote{The camera also disposes of "free-run mode".}.

The camera is equipped with AF Nikkor 50mm f1.8D lens. The lens suits our needs for multiple reason. It is relatively inexpensive, simple to use and has good depth-of-field control with aperture ranging from f/1.8 up to f/22.

\subsubsection*{Camera Mounting}
In order to acquire images with line camera either the object or camera has to move in smooth direct line. We decided to move the camera for various reason. The main reason is the human error. People can easily place a hand on a firm stable surface and keep the hand as it is for a long period of time, in our case no more than 30 seconds.

The camera is mounted on a slide rail with stepper motor with the platform facing down in order to mount the camera and light as seen in figure \ref{fig:rail}. The slide rail itself is mounted on an aluminium scaffolding which raises the rail to the desired height.

\begin{figure}[h]
    \label{fig:rail}
    \includegraphics[width=\linewidth]{rail.jpg}
    \caption{Camera mount setup. ** tohle se jeste jednou vyfoti **}
\end{figure}

\subsubsection*{Slide Rail \& Stepper Motor}
We proposed for straight and smooth movement in one axis a slide rail controlled by a stepper motor. The stepper motor itself is controlled by Leadshine EM705 Digital Stepper Drive which is controlled by pulse width modulation generated by Arduino UNO with our custom code.

\subsection{Hand Placement}
The hand is placed in the centre of the scaffolding on a prepared foundation which aligns and spreads the fingers in order to always capture the same hand geometry. The hand stays put during the whole image acquisition procedure which minimizes the human error.

\begin{figure}[h]
    \label{fig:rail}
    \includegraphics[width=0.5\linewidth]{rail.jpg}
    \includegraphics[width=0.5\linewidth]{rail.jpg}
    \caption{Tu bude neci ruka s deskou ** tohle se jeste jednou vyfoti **}
\end{figure}


\subsection{Slide Rail Control}
% TODO: nastavenie kontroleru - popisat problemy zo sekanim
% ako sme dospely k poctu sekund tam a s5 / pripadne vypocet vzorcom
Pohyb vozíka po koľajnici spočíva z jednosmerného pohybu stálou rýchlosťou vpravo, zastavení a opačného pohybu do pôvodnej pozície. \\
Pre pohyb vozíka je potrebná kontrola ovládacej jednotky motora, ktorý ovláda pohyb vozíka po koľajnici. Tento motor je ovládaný prostredníctvom Arduino UNO, ktoré prostredníctvom GPIO povoľuje pohyb a nastavuje smer pohybu. Pre ovládanie tejto fukcionality sa využíva ako kontrolér Raspberry Pi 3. Na raspberry beží program ktorý používa python program pre ovládanie GPIO pinov ktoré sú pripojené ku Arduino. Konkrétne sa využívajú výstupné GPIO piny:
\begin{itemize}
    \item PIN 18 - PWM
    \item PIN  5 - smer pohybu
    \item PIN  6 - povolenie pohybu
\end{itemize}


\subsection{Light Control}
\subsection{Camera Control}
%TODO: ako je ovladana kamera a prečo je to takto
% aka je vykonnost kamery a ake nastavenia prebehli
% ake je nastavenie osvetlenia -> pripadne samostatna sekcia o osvetleni


\section{Software}
Hardware solution for proposed device consists from three main parts: moving slide on rail, camera and light.
Each of these parts needs to be controlled in order to scan images of hand geometry.
All of the software as well as webserver storing results runs on control module - Raspberry Pi.

\label{sec:sw}
\subsection{Used Software}
Software is implemented in python language for its ease of use and comprehension. With all necessary software running
on Raspbery Pi minimum setup is necessary.

Slide platform is controlled through serial bus by writing to GPIO pins direction and enable signals.
Light is controlled via its XLC4 controller through RS232 interface.
Light status is controlled through commands \texttt{LC A 1} and \texttt{LC A 0} for turning it on/off respectively.
\todo[inline]{camera control, skladanie obrazkov}
\todo[inline]{referencie na zdrojove kody}
\todo[inline]{zoznam dependencies, instalacia}

\subsection{Acquiring Images}
To acquire images simple setup is necessary. Assembly and connection of all hardware parts is necessary with individual
parts connected to Raspberry Pi control module according to schematics. Raspberry control module needs to be setup and
software installed.
When Raspberry is connected to internet image acquisition can be done through remote access without need for whole desktop.
Firstly hand is placed on designated platform and then software is run. Firstly lights are switched on, then camera starts
recording and slide moves along the rail till it completes the scan. After scan slide moves to initial position and device
is immediately ready for another scan.




\section{Hand Geometry Database}
\label{sec:db}
% TODO kam sa ukladaju snimky, ako sa k nim pristupuje
% ako dlho trva scan snimku a aka je kvalita snimkov
% preco sa ukladaju tam a ako je mozno dosiahnut rozsirenie DB




\section{Achieved Results}
\label{sec:res}
Solution described in this report was assembled and implemented with excellent results. Integration of all hardware parts and software control results
in images being scanned very quickly in great resolution. During testing it was proved to be effective to not only scan images
of hand geometry but also fingerprints which leads to general purpose sensoric solution for capturing multiple biometric data. Furthermore,
solution is able to function with very limited resources necessary and provides instant access to scanned images.

\todo[inline]{rychlost, kvalita, mnostvo snimkov}
\begin{figure}[ht!]
    \label{fig:hand-detail}
    \centering
    \includegraphics[width=0.5\linewidth]{hand-detail.jpg}
    \caption{Dorsal detail of scanned hand}
\end{figure}

\section{Summary}
\label{sec:sum}
\todo[inline]{preciznost HW, vylepsenie - antivibracna podlozka, 3d geometria, distancne stlpky a palcement ruky}
As previously dicussed in \ref{sec:res} proposed solution si able to quickly and accurately provide 2D hand geometry in great resolution. Solution was
used to create initial image database with emphasisis on ease of use and precision so this database can be quickly expanded.
Solution proved to be effective to not only capture 2D hand geometry but also capture other biometric data such as fingertip scans. This results in
ability to make multiple models from data and design efficient classification and identification algorhitms.
\end{document}
