\documentclass[11pt,a4paper]{article}
\usepackage[left=2cm,text={17cm,25cm},top=2.5cm]{geometry}
\usepackage[T1]{fontenc}
\usepackage[english]{babel}
\usepackage[utf8]{inputenc}
\usepackage{url}
\usepackage{graphicx}
\usepackage{pdfpages}
\graphicspath{ {img/} }

\begin{document}

\begin{center}
	\LARGE{Biometric Systems}\\
	\Large{Capturing of 2D Hand Geometry With Line Camera}
	\vspace{0.5cm}

    \begin{centering}
    \small{
        Petr Stehlík <xstehl14@stud.fit.vutbr.cz>\\
        Marek Beňo <xbenom01@stud.fit.vutbr.cz>\\
        Richard Wolfert <xwolfe00@stud.fit.vutbr.cz>\\
        }
    \end{centering}

	\vspace{0.2cm}

\end{center}

\section{Assignment}
The task in hand is to assemble a mechanical device and create software for acquiring 2D hand geometry via a line camera. A solution for mounting the camera and hand placement should be proposed in the following sections for best geometry acquisition and image reconstruction.

The document is structured as follows: in section \ref{sec:hw} we describe the hardware part of the project, what hardware was used and we propose a solution for camera mount and hand placement. In section \ref{sec:sw} the software needed for camera control and image reconstruction is presented. In section \ref{sec:db} we detail the created hand geometry database and how we acquired it and in last two sections \ref{sec:res} and \ref{sec:sum} we sum up the proposed hardware and software solution and what further work can be done.

\section{Hardware}
\label{sec:hw}
\subsection{Used Hardware}
\subsection{Hardware Setup}
% TODO: čo bolo treba vytvoriť na sprovoznenie
% zapojenie arduino, raspberry
% popísať prečo je potreba arduino a púrečo je pi zlé (GPIO 3.3V vs 5V)
% schema konstrukcie / foto

\subsection{Camera Mount}
\subsection{Hand Placement}

\subsection{Rail Control}
% TODO: nastavenie kontroleru - popisat problemy zo sekanim
% ako sme dospely k poctu sekund tam a s5 / pripadne vypocet vzorcom
Pohyb vozíka po koľajnici spočíva z jednosmerného pohybu stálou rýchlosťou vpravo, zastavení a opačného pohybu do pôvodnej pozície. \\
Pre pohyb vozíka je potrebná kontrola ovládacej jednotky motora, ktorý ovláda pohyb vozíka po koľajnici. Tento motor je ovládaný prostredníctvom Arduino UNO, ktoré prostredníctvom GPIO povoľuje pohyb a nastavuje smer pohybu. Pre ovládanie tejto fukcionality sa využíva ako kontrolér Raspberry Pi 3. Na raspberry beží program ktorý používa python program pre ovládanie GPIO pinov ktoré sú pripojené ku Arduino. Konkrétne sa využívajú výstupné GPIO piny:
\begin{itemize}
    \item PIN 18 - PWM
    \item PIN  5 - smer pohybu
    \item PIN  6 - povolenie pohybu
\end{itemize}


\subsection{Light Control}
\subsection{Camera Control}
%TODO: ako je ovladana kamera a prečo je to takto
% aka je vykonnost kamery a ake nastavenia prebehli
% ake je nastavenie osvetlenia -> pripadne samostatna sekcia o osvetleni


\section{Software}
\label{sec:sw}
\subsection{Used Software}
\subsection{Acquiring Images}
\subsection{Detecting Geometry}
% TODO: aky algoritmus sme zvolili v čom sme to implementovali a prečo

\section{Hand Geometry Database}
\label{sec:db}
% TODO kam sa ukladaju snimky, ako sa k nim pristupuje
% ako dlho trva scan snimku a aka je kvalita snimkov
% preco sa ukladaju tam a ako je mozno dosiahnut rozsirenie DB

\section{Achieved Results}
\label{sec:res}

\section{Summary}
\label{sec:sum}

\end{document}
