\section{Software}
Hardware solution for proposed device consists from three main parts: moving slide on rail, camera and light.
Each of these parts needs to be controlled in order to scan images of hand geometry.
All of the software as well as webserver storing results runs on control module - Raspberry Pi.

\label{sec:sw}
\subsection{Used Software}
Control software is implemented in python language for its ease of use and comprehension. With all necessary software running
on Raspbery Pi minimum setup is necessary.
When acessing system through Raspberry Pi control module user script \texttt{rail-control/capture.py} handles all
neccessary communication with hardware parts while \texttt{src/capture} handles image capture using
basler pylon 5 library and image reconstruction using OpenCV library. During testing, these \texttt{C++} libraries proved to be more
reliable then their python counterparts and so they were interfaced with rest of the codebase.

\subsection{Acquiring Images}
To acquire images simple setup is necessary. Assembly and connection of all hardware parts is necessary with individual
parts connected to Raspberry Pi control module according to schematics. Raspberry control module needs to be setup and
software installed.
When Raspberry is connected to internet image acquisition can be done through remote access. Initially, hand should be placed on
designated platform and then software is run. Firstly lights are switched on, then camera starts
recording and slide moves along the rail till it completes the scan. Image is then processed and saved in non-volatile memory using ".jpg" format.
The light is then turned off, which indicates that image was successfully processed and saved. Finally, scan slide moves to initial position and
device is immediately ready for another scan.
