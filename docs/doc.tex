\documentclass[11pt,a4paper]{article}
\usepackage[left=2cm,text={17cm,25cm},top=2.5cm]{geometry}
\usepackage[T1]{fontenc}
\usepackage[english]{babel}
\usepackage[utf8]{inputenc}
\usepackage{url}
\usepackage{graphicx}
\usepackage{pdfpages}
\graphicspath{ {img/} }

\begin{document}

\begin{center}
	\LARGE{Biometrické systémy}\\
	\Large{Snímání 2D geometrie ruky řádkovou kamerou}
	\vspace{0.5cm}

	Petr Stehlík, Marek Beňo, Richard Wolfert

	\vspace{0.2cm}

\end{center}

\section{Cieľ projektu}
Cieľom tohoto projektu je vytvoriť mechanické zariadenie a pripraviť softvérové riešenie umožňujúce snímať 2D scan ruky. Scan ruky je zaobstaraný pomocou riadkovej kamery mechanicky pripevnenej na pohybujúci sa vozík, ktorý sa pohybuje ponad snímanú oblasť. Zosnímaný scan ruky je následné verifikovaný proti existujúcej databázi snímkov pomocou implementovaného verifikačného algoritmu.

\section{Konstrukce zařízení pro snímání 2D geometrie ruky}
% TODO: svetlo?
Zariadenie spočíva z masívnej konštrukcie, kolajnice s vozíkom, zdrojov napájania, ovládacích prvkov a kamery. 
% TODO: čo bolo treba vytvoriť na sprovoznenie
% zapojenie arduino, raspberry
% popísať prečo je potreba arduino a púrečo je pi zlé (GPIO 3.3V vs 5V)
% schema konstrukcie / foto 

\subsection{Ovládanie pohybu vozíka}
% TODO: nastavenie kontroleru - popisat problemy zo sekanim
% ako sme dospely k poctu sekund tam a s5 / pripadne vypocet vzorcom
Pohyb vozíka po koľajnici spočíva z jednosmerného pohybu stálou rýchlosťou vpravo, zastavení a opačného pohybu do pôvodnej pozície. \\
Pre pohyb vozíka je potrebná kontrola ovládacej jednotky motora, ktorý ovláda pohyb vozíka po koľajnici. Tento motor je ovládaný prostredníctvom Arduino UNO, ktoré prostredníctvom GPIO povoľuje pohyb a nastavuje smer pohybu. Pre ovládanie tejto fukcionality sa využíva ako kontrolér Raspberry Pi 3. Na raspberry beží program ktorý používa python program pre ovládanie GPIO pinov ktoré sú pripojené ku Arduino. Konkrétne sa využívajú výstupné GPIO piny:
\begin{itemize}
    \item PIN 18 - PWM
    \item PIN  5 - smer pohybu
    \item PIN  6 - povolenie pohybu
\end{itemize}

\subsection{Ovládanie kamery}
%TODO: ako je ovladana kamera a prečo je to takto
% aka je vykonnost kamery a ake nastavenia prebehli
% ake je nastavenie osvetlenia -> pripadne samostatna sekcia o osvetleni

\section{Implementace algoritmu pre rekonštrukciu obrazu}
% TODO: aky algoritmus sme zvolili v čom sme to implementovali a prečo

\section{Databáze snímku}
% TODO kams a ukladaju snimky, ako sa k nim pristupuje
% ako dlho trva scan snimku a aka je kvalita snimkov
% preco sa ukladaju tam a ako je mozno dosiahnut rozsirenie DB

\section{Algoritmus verifikace snímku}

\section{Záver}

\end{document}
